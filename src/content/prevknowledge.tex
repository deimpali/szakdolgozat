%----------------------------------------------------------------------------
\chapter{Előismeretek}
%----------------------------------------------------------------------------

%\section{Részleges modellek fajtái}
%\subsection{May}
%\subsection{Abs}
%\subsection{Var}
%\subsection{OW}
\section{Bemutatás egy példa segítségével}
%pl osztálydiagram

A részleges modellezést legegyszerűbben egy gyakorlati példán lehet bemutatni. Vegyük példának az UML osztálydiagramját. Az UML (Unified Modeling Language) egy szabványos, objektumorientált modellezési nyelv, ami a tervezést, fejlesztést és egyéb folyamatokat segíti. Ezt a nyelvet informatikában és egyéb üzleti területeken egyaránt alkalmazzák. Magában foglal több diagram típust is.
\par
Az osztálydiagram egy strukturális diagram, segítségével egy rendszerben előforduló objektumokat, azok tulajdonságait és kapcsolatait lehet modellezni. Az objektumok téglalapokkal vannak jelezve, a hozzájuk tartozó attribútumok a téglalapon belül helyezkednek el és az elemek összekötésével a kapcsolatokat lehet jelezni.
\par
Az alábbi példában (lásd \autoref{jarmu}) egy járműnek az osztálydiagramja látható. A járműnek vannak attribútumai és tartalmaz ajtót, kereket és üléseket. Továbbá van neki egy meghajtása, ami egy absztrakt osztály. Ezek ugyan azon attribútumokkal van ellátva: id, típus, név. A meghajtásnak két leszármazottja van a benzinmotoros és a villanymotoros meghajtás. A benzinmotor tulajdonsága a hengerek száma a villanymotor tulajdonsága pedig az akkumulátor kapacitása.

\begin{figure}[!ht]
	\centering
	\includegraphics[width=130mm]{figures/vehicle.pdf}
	\caption{Jármű egyszerűsített osztálydiagramja} 
	\label{jarmu}
\end{figure}

\section{Részleges modell}
Részleges modell segítségével lehetőség van tervezői döntések dokumentálására. Általában a modellező eszközökkel csak érvényes modelleket lehet készíteni. Így a modell készítője belekényszerül olyan döntések meghozatalába, amiket csak később kéne meghozni. Emiatt elveszhetnek tervezői döntések. Részleges modellben lehetőség van ezeket a kétséges helyzeteket már a modellezés szintjén kezelni. Így a modell nem csupán strukturális információt tartalmaz, hanem a részlegességéről is, tehát hogy teljesen specifikált a modell vagy sem. Az előbbi példát tekintve, lehetséges az hogy a járműnek egyáltalán nem szeretnénk ajtót, mert például egy motort szeretnénk modellezni. Ekkor feleslegessé válik teljesen az ajtó jelenléte a modellben. Erről információt az UML szabályai szerint nem tudunk tárolni. Vagy feljegyezzük a lehetséges változtatást, vagy pedig létrehozunk egy másik diagramot, amibe van a járművön ajtó és egy olyat amiben nincs. Ezek egyike sem tűnik jó megoldásnak. Egy ilyen picike diagram esetén még akár átlátható de egy nagyobb, akár több száz elemből álló modell esetén, ha máshol is van ilyen kétség a végleges modellel kapcsolatban, akkor már kezelhetetlenné válik. Például 5 ilyen döntés esetében 2\^5 darab diagramot kéne párhuzamosan fejleszteni.
\par
Lehetőség van a modell finomítására is (lásd \autoref{finomit}). Finomítás során a részleges modellből kikerülnek bizonytalan elemek. Ez azt jelenti, hogy a modellben jelzett részlegesség mértéke csökkenthető és ennek eredményeképpen véges számú finomítás után a modellből teljesen eltűnnek a részlegességek.

\begin{figure}[!ht]
	\centering
	\includegraphics[width=130mm]{figures/finom.pdf}
	\caption{Finomítás menete} 
	\label{finomit}
\end{figure}

\subsection{Szintaktika}
Részleges modellek esetén annotációkkal lehet megjelölni a modellt, illetve annak elemeit. Modellezés során 4 fajta részlegességet jelölhetünk meg. 
Annotációk

\subsection{Szemantika}
\subsubsection{May}
Annotációkkal láthatjuk el a modellt, az alapján, hogy egy modellelem biztosan benne lesz a végleges modellben, vagy pedig még bizonytalan a léte. \textsf{’M’} May exist (lehet, hogy benne lesz a végleges modellben, de lehet, hogy nem), \textsf{’E’} Must exist (biztosan benne lesz a végleges modellben). A modell finomításával lépésről lépésre egyre kevesebb ’M’ lesz.  \textsf{’M’} az vagy átvált \textsf{’E’}–re, vagy teljesen kikerül a modellből. Akkor tekinthető a modell véglegesnek, ha már nem szerepel benne \textsf{’M’}-el megjelölt elem.
\par
Tegyük fel, hogy nem tudjuk milyen járművet akarunk még modellezni a kezdeti fázisban. Ezért a kiindulási objektummodellben elláttuk May annotációkkal a járműnek az ajtó elemét. Így finomítás során ez az elem később eltűnhet de akár meg is maradhat. Az a jármű aminek nincs ajtaja lehet akár egy motor, a két ajtós változat pedig egy autó. Erre példa az alábbi diagramrészlet (lásd \autoref{may}).

\begin{figure}[htp]
	\centering
	\includegraphics[width=130mm]{figures/may.pdf}
	\caption{May részlegesség feloldása} 
	\label{may}
\end{figure}

\subsubsection{Abs}
Ennek is két különböző annotálási módja lehet: \textsf{’P’}, Particular (egyedi elem) és \textsf{’S’}, azaz Set (egy vagy több elemet jelölhet). Particular az olyan elem, ami már biztosan benne lesz a végleges modellben, azonban a Set olyan elemet jelöl, ami lehet, hogy a végleges modellben több elemként fog megjelenni. Finomítások során az \textsf{’S’}-el megjelölt tagokat szétbontjuk több részre vagy meghagyhatjuk egyedi elemként, de a végleges modellben már csak particular, egyedi elemek szerepelhetnek.
\par
Diagram kezdeti fázisában még nincs eldöntve, hogy az ülés az egyedi elem vagy sem, ezért meg van jelölve egy \textsf{’S’} annotációval. Finomítás során lehetséges, hogy az ülést megtartjuk eredeti formájában. A másik lehetőség viszont az, hogy szétbontjuk első illetve hátsó ülésre. Ebben az esetben ugyan azok a tulajdonságok lesznek meg mind a két ülésben de mégis modell szempontjából külön kezelendőek. Erre példa az alábbi diagramrészlet (lásd \autoref{abs}).

\begin{figure}[htp]
	\centering
	\includegraphics[width=130mm, keepaspectratio]{figures/abs.pdf}
	\caption{Abs részlegesség feloldása} 
	\label{abs}
\end{figure}

\subsubsection{Var}
Kétféleképpen lehet annotálni az elemeket: \textsf{’C’} Constant (konstans elem) és \textsf{’V’} Variable (változó elem). Bizonyos értelemben az Abs fordítottjának lehet tekinteni. Felveszünk több elemet, ami lehet, hogy későbbiekben összeolvasztunk egy elemmé, tehát fenn áll a lehetősége annak, hogy két elem megegyezik. Amikor elkezdünk egy modellt, akkor nem biztos, hogy meg tudjuk mondani elemekről, hogy azok a későbbiekben azonosak-e vagy különbözőek. A végleges modellben már csak konstans elemek lehetnek.
\par
Itt a példában (lásd \autoref{var}) látható, hogy a kezdeti állapotban még két külön kereke van a járműnek egyik kerekén kék dísztárcsa van a másik kereke pedig széles felnivel rendelkezik. Ezek meg lettek jelölve \textsf{'V'} annotációval. Finomítás után ez megmaradhat ilyen különálló formában, de akár összeolvaszthatjuk ezt a két kereket és a végeredményben egy kerék marad, ami mind a kettő kerék tulajdonságát magában hordozza.

\begin{figure}[htp]
	\centering
	\includegraphics[width=130mm, keepaspectratio]{figures/var.pdf}
	\caption{Var részlegesség feloldása} 
	\label{var}
\end{figure}

\subsubsection{OW}
Modellezés folyamán lehet megjelölni azt, hogy a modell már végleges-e vagy sem, tehát adhatunk-e új elemeket a modellhez vagy nem. Ez a részlegesség nem a modell elemeire vonatkozik, hanem a teljes modellről árul el információt. 
