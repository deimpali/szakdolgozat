%----------------------------------------------------------------------------
\chapter{\bevezetes}
%----------------------------------------------------------------------------

\section{Témamegjelölés}
Modellvezérelt tervezésnek nagy szerepe van az informatikában egy szoftver vagy rendszer létrehozásánál. Modellek segítségével sokkal strukturáltabban és megbízhatóbban lehet tervezni. A modellek helyességét automatikusan ellenőrizni és a modellből kódot generálni is lehet. 
%Ehhez a rengeteg eszköz áll rendelkezésünkre, viszont ezek nem mindenre nyújtanak megoldást. 
\section{Problémafelvetés}
Mai modellező eszközökkel általában nem lehetséges, hogy részleges, vagy hibás modelleket kezeljünk, elmentsünk. A modell készítése közben lehetnek döntések, amikkel nem is feltétlen szükséges foglalkozni az adott tervezési szakaszban, mégis ahhoz, hogy a modell érvényes legyen rákényszerül az ember. Ezeket az elemeket célszerű lenne valamilyen módon megjelölni és a szerkesztést folytatni. Előfordulhat hogy a modellben egy elem megléte, vagy a multiplicitása kétséges. Esetleg olyan, hogy nem lehet eldönteni egy elemről hogy az melyik másikkal áll kapcsolatban.
\section{Célkitűzés}	
Dolgozatom célja, olyan általános célú modell elkészítése, aminek segítségével lehet részleges, vagy hiányos modelleket is ábrázolni. Például, előfordulhat olyan, hogy nem tudjuk eldönteni egy attribútumról, hogy melyik elemhez tartozik. Ilyet általában a modellező eszközök nem támogatnak. Célszerű lenne egy olyan általános módszert kitalálni, amivel lehetséges az ilyen és ehhez hasonló esetek megjelölése, kezelése.
\section{Kontribúció}
Kutatásom során megismerkedtem a részleges modellezés leglényegesebb aspektusaival. Létrehoztam egy olyan általános célú modellező nyelvet, ami alapján lehetséges részleges modellek készítése. Ezután elkészítettem egy olyan vizuális szerkesztőfelületet, aminek a segítségével parciális modell példányokat lehet készíteni és szerkeszteni.  

\section{Hozzáadott érték}	
A Szerkesztő nem csupán egy megjelenítő eszközt biztosít a részleges modellnek, hanem más funkciókkal is ellátja azt. A modell finomítására is lehetőséget ad, ami a részleges modellek szerkesztésének egy fontos funkciója. Ezáltal gyorsabb a prototipizálás és a tervezési döntések dokumentálása.
%Ennek jelentését és jelentőségét a dolgozat folyamán részletesen kifejtem.

\section{Dolgozat felépítése}
Előismeretek részben (lásd \autoref{chapter:preknowledge}) először egy konkrét példát mutatok egy modellre. Későbbiekben ezen a példán keresztül mutatom be a részleges modellezés lényegét. A parciális modellezésnek először a szintaktikáját tárgyalom, majd a hozzá tartozó szemantikát ismertetem. Ezekben a részekben a részlegesség négy fajtáját emelem ki: May, Var, Abs, OW. Áttekintés fejezetben (lásd \autoref{chapter:overview}) a konkrét feladat tervét mutatom be egy ábra segítségével. Ismertetem a modell felépítését illetve a szerkesztőfelület lehetőségeit. Ezt követően a megvalósítási folyamat előzményeként kitérek a technológiákra, amikre szükség volt a megvalósításhoz (lásd \autoref{chapter:realization}). A szerkesztőfelület dokumentálása és működésének bemutatását követően végül a továbbfejlesztési lehetőségekről is lesz szó (lásd \autoref{chapter:summary}). 
