%----------------------------------------------------------------------------
\chapter{\bevezetes}
%----------------------------------------------------------------------------

\section{Témamegjelölés}
Modell-vezérelt tervezésnek nagy szerepe van az informatikában egy szoftver vagy rendszer létrehozásánál. Modellek segítségével sokkal jobban, strukturáltabban és megbízhatóbban lehet tervezni. A modellek helyességét automatikusan ellenőrizni és a modellből kódot generálni is lehet. Ehhez a rengeteg eszköz áll rendelkezésünkre, viszont ezek nem mindenre nyújtanak megoldást. 
\section{Problémafelvetés}
Mai modellező eszközökkel általában nem lehetséges, hogy részleges, vagy hibás modelleket kezeljünk, elmentsünk. A modell készítése közben lehetnek döntések, amikkel nem is feltétlen szükséges foglalkozni az adott tervezési szakaszban, mégis ahhoz, hogy a modell érvényes legyen rákényszerül az ember. Ezeket az elemeket célszerű lenne valamilyen módon megjelölni és a szerkesztést folytatni. Előfordulhat hogy a modellben egy elem megléte, vagy a multiplicitása kétséges. Esetleg, olyan, hogy nem lehet eldönteni egy elemről hogy az melyik másikkal áll kapcsolatban.
\section{Célkitűzés}	
Dolgozatom célja, olyan általános célú modell elkészítése, aminek segítségével lehet részleges, vagy hiányos modelleket is ábrázolni. Például, előfordulhat olyan, hogy nem tudjuk eldönteni egy attribútumról, hogy melyik elemhez tartozik. Ilyet általában a modellező eszközök nem támogatnak. Célszerű lenne egy olyan általános módszert kitalálni, amivel lehetséges az ilyen és ehhez hasonló esetek megjelölése, kezelése.
\section{Kontribúció}
Kutatásom során megismerkedtem a részleges modellezés leglényegesebb aspektusaival. Létrehoztam egy olyan modellt, ami alapján lehetséges részleges modellek készítése. Ezután elkészítettem egy olyan vizuális szerkesztőfelületet, aminek a segítségével részleges modell példányokat lehet készíteni.  

\section{Hozzáadott érték}	

\section{Dolgozat felépítése}

