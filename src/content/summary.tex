%----------------------------------------------------------------------------
\chapter{Összefoglalás és továbbfejlesztési lehetőségek}\label{chapter:summary}
%----------------------------------------------------------------------------
A részleges modellek vizsgálatát rengeteg más szemszögből is meg lehet közelíteni. Lehetséges a szerkesztőt továbbfejleszteni, új funkciókkal és validációkkal bővíteni. 
Jelenleg a modellező eszközben tudunk készíteni parciális modelleket, de már meglévő modellek szerkesztése részleges modellként nem lehetséges. Erre lehetne készíteni egy programot, ami szerkeszteni kívánt modell minden elemét megfelelteti egy parciális modellbeli elemmel. Így például a dolgozatban szereplő jármű (lásd \autoref{jarmu}) minden objektumát "becsomagolhatjuk" agy részleges modellhez tartozó objektumba. Ezután ez az elkészített szerkesztőben módosítható. Amennyiben nem kívánunk változtatni a modell egy hasonló programmal visszaalakítható lehet. Ezen a szálon tovább haladva lehetne részlegességet tartalmazó modellből is generálni az importálthoz hasonló modellt. A generáló program többféle modellt készíthet a parciális modellből, figyelembe véve, hogy milyen részlegességeket tartalmaz.
