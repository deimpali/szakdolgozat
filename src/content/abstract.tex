\pagenumbering{roman}
\setcounter{page}{1}

\selecthungarian

%----------------------------------------------------------------------------
% Abstract in Hungarian
%----------------------------------------------------------------------------
\chapter*{Kivonat}\addcontentsline{toc}{chapter}{Kivonat}

%Jelen dokumentum egy diplomaterv sablon, amely formai keretet ad a BME Villamosmérnöki és Informatikai Karán végző hallgatók által elkészítendő szakdolgozatnak és diplomatervnek. A sablon használata opcionális. Ez a sablon \LaTeX~alapú, a \emph{TeXLive} \TeX-implementációval és a PDF-\LaTeX~fordítóval működőképes.

Informatika világában manapság egyre elterjedtebb rendszerek és szoftverek fejlesztésénél, hogy modellező eszközöket használunk. Ezen eszközök hatékonyabbá teszik a tervezési munkálatokat, azonban többségük nem biztosít lehetőséget arra, hogy hiányos, vagy hibás modelleket készítsünk. A szoftver általában már a tervezés alatt is megkövetelik azt, hogy a modellünk jólformált és teljes legyen. Ez sokszor olyan döntések meghozatalát kényszeríti ki, amivel nem szeretnénk a tervezés azon szakaszában foglalkozni. Ilyen esetben a legtöbb, amit tehetünk, hogy megjegyzéseket írunk fel. Jó lenne egy olyan eszköz, ami a modellezés absztrakciós szintjén képes kezelni az ilyen döntéseket. 
Egyre többen foglalkoznak ezzel a kérdéssel, és sokféle megközelítés született már a problémára. Így került előtérbe a parciális modell és a MAVO (May-Arbitrary-Variable-Open word) absztrakció is. 
\par
Dolgozatomban egy olyan általános célú modellező környezetet mutatok be, aminek segítségével lehetséges részleges modelleket létrehozni és szerkeszteni. Az eszköz a MAVO absztrakciót támogatja. Ez úgy jelenik meg, hogy annotációkkal lehet felruházni a bizonytalan elemeket a modellben. Illetve egy esetben, az 'Open world' esetében magát a modellt lehet megjelölni. Az annotációk nem csupán jelzés értékűek, hanem funkcionalitás is társul hozzájuk. A modellező eszköz biztosít az annotációk feloldására automatizált megoldást, ezt finomításnak nevezik. Ezáltal csökkenthető a modell részlegessége, ami végül egy olyan modellhez vezethet, amiben egyáltalán nincsen annotáció, tehát teljes.


\vfill
\selectenglish


%----------------------------------------------------------------------------
% Abstract in English
%----------------------------------------------------------------------------
\chapter*{Abstract}\addcontentsline{toc}{chapter}{Abstract}

%This document is a \LaTeX-based skeleton for BSc/MSc~theses of students at the Electrical Engineering and Informatics Faculty, Budapest University of Technology and Economics. The usage of this skeleton is optional. It has been tested with the \emph{TeXLive} \TeX~implementation, and it requires the PDF-\LaTeX~compiler.


\vfill
\selectthesislanguage

\newcounter{romanPage}
\setcounter{romanPage}{\value{page}}
\stepcounter{romanPage}